\documentclass[book.tex]{subfiles}
\begin{document}

\chapter{Rank-N Types}

\section{Introduction}

Sometimes Haskell's default notion of polymorphism simply isn't polymorphic
\emph{enough.} To demonstrate, consider a contrived function which takes the
\hs{id :: a -> a} as an argument, and applies it to the number \hs{5}.  Our
first attempt might look something like this:

\snip{RankN}{brokenApply}

The reasoning here is that because \hs{id} has type \ty{a -> a},\;
\hs{applyToFive} should have type \ty{(a -> a) -> Int}. Unfortunately,
Haskell disagrees with us when we try to compile this.

\begin{alltt}
<interactive>:2:32: error:
    • Couldn't match expected type ‘a’ with actual type ‘Int’
\end{alltt}

We can't apply \hs{f} to \hs{5} because, as the error helpfully points out,
\ty{a} is not an \ty{Int}. Recall that under normal circumstances, the
\emph{caller} of a polymorphic function is responsible for choosing which
concrete types those variables get. The signature \ty{(a -> a) -> Int} promises
that \hs{applyToFive} will happily take any function which returns the same type
it takes.

We wanted \hs{applyToFive} to only be able to take \hs{id} as a parameter, but
instead we've written a function which (if it compiled) would happily take any
\defn{endomorphism}.\footnote{Functions which take and return the same type.
For example, \hs{not :: Bool -> Bool}, \;\hs{show @String :: String -> String} and
\hs{id :: a -> a} are all endomorphisms, but \hs{words :: String -> [String]} is
not.} Because the choice of \ty{a} is at the mercy of the caller, Haskell has no
choice but to reject the above definition of \hs{applyToFive}---it would be a
type error to try to apply \hs{5} to \hs{not}, for example.

And so we come to the inevitable conclusion that, as is so often the case, the
compiler is right and we (or at least, our type) is wrong. The type of
\hs{applyToFive} simply doesn't have enough polymorphism. But why not, and what
can we do about it?

The discrepancy comes from a quirk of Haskell's syntax. By default, the language
will automatically quantify our type variables, meaning that the type signature
\ty{a -> a} is really syntactic sugar for \ty{forall a. a -> a}. By enabling
\ext{RankNTypes} we can write these desugared types explicitly. Comparing
\hs{id} and \hs{applyToFive} side-by-side is revealing.

\snip{RankN}{id}
\snip{RankN}{explicitBrokenApply}

Recall that we intended to give the type of \hs{id} for the first parameter of
\hs{applyToFive}. However, due to Haskell's implicit quantification of type
variables, we were lead astray in our attempts. This explains why
\hs{applyToFive} above didn't compile.

The solution is easy: we simply need to move the \hs{forall a.} part inside of
the parentheses.

\snip{RankN}{applyToFive}

\begin{dorepl}{RankN}
applyToFive id
\end{dorepl}

In this chapter we will dive into what rank-n types are and what their more
interesting uses can do for us.


\section{Ranks}

The \ext{RankNTypes}\index{RankNTypes} is best thought of as making polymorphism
\emph{first-class}. It allows us to introduce polymorphism anywhere a type is
allowed, rather than only on top-level bindings.\footnote{And let-bound
expressions, though this polymorphism is usually invisible to the everyday
Haskell programmer.}

While relaxing this restriction is ``obviously a good thing'', it's not without
its sharp edges. In general, type inference is undecidable in the presence of
higher-rank polymorphism.\footnote{Theoretically it's possible to infer types
for rank-2 polymorphism, but at time of writing GHC doesn't.} Code that doesn't
interact with such things need not worry, but higher-rank polymorphism always
requires an explicit type signature.

But what exactly \emph{is} a rank?

In type-theory lingo, the \defn{rank} of a function is the ``depth'' of its
polymorphism. A function that has no polymorphic parameters is rank 0. However,
most, if not all, polymorphic functions you're familiar with---\hs{const :: a ->
b -> a}, \;\hs{head :: [a] -> a},\; etc---are rank 1.

The function \hs{applyToFive} above is rank 2, because its \hs{f} parameter
itself is rank 1.  In principle there is no limit to how high rank a function
can be, but in practice nobody seems to have gone above rank 3. And for good
reason---higher-rank functions quickly become unfathomable. Rather than
explicitly counting ranks, we usually call any function above rank-1 to be
\defn{rank-n} or \defn{higher rank}.

The intuition behind higher-rank types is that they are \emph{functions which
take callbacks}. The rank of a function is how often control gets ``handed
off''. A rank-2 function will call a polymorphic function for you, while a
rank-3 function will run a callback which itself runs a callback.

Because callbacks are used to transfer control from a called function back to
its calling context, there's a sort of a seesaw thing going on. For example,
consider an (arbitrarily chosen) rank-2 function \ty{foo :: forall r. (forall a.
a -> r) -> r}. As the caller of \hs{foo}, we are responsible for determining the
instantiation of \ty{r}. However, the \emph{implementation} of \hs{foo} gets to
choose what type \ty{a} is. The callback you give it must work for whatever
choice of \ty{a} it makes.

This is exactly why \hs{applyToFive} works. Recall its definition:

\snip{RankN}{applyToFive}

Notice that the implementation of \hs{applyToFive} is what calls \hs{f}. Because
\hs{f} is rank-1 here, \hs{applyToFive} can instantiate it at \ty{Int}. Compare
it with our broken implementation:

\snip{RankN}{explicitBrokenApply}

Here, \hs{f} is rank-0 because it is no longer polymorphic---the caller of
\ty{applyToFive} has already instantiated \ty{a} by the time \ty{applyToFive}
gets access to it---and as such, it's an error to apply it to \hs{5}. We have no
guarantees that the caller decided \ty{a \tyeqSpace Int}.

By pushing up the rank of \hs{applyToFive}, we can delay who gets to decide the
type \ty{a}. We move it from being the caller's choice to being the
\emph{callee's} choice.

Even higher-yet ranks also work in this fashion. The caller of the function and
the implementations seesaw between who is responsible for instantiating the
polymorphic types. We will look more deeply at these sorts of functions later.

\section{The Nitty Gritty Details}

It is valuable to formalize exactly what's going on with this rank stuff. More
precisely, a function gains higher rank every time a \hs{forall} quantifier
exists on the left-side of a function arrow.

But aren't \hs{forall} quantifiers \emph{always} on the left-side of a function
arrow? While it might seem that way, this is merely a quirk of Haskell's syntax.
Because the \hs{forall} quantifier binds more loosely than the arrow type
\hs{(->)}, the everyday type of \hs{id},

\snip{RankN}{forall1}

has some implicit parentheses. When written in full:

\snip{RankN}{forall2}

it's easier to see that the arrow is in fact captured by the \hs{forall}.
Compare this to a rank-\emph{n} type with all of its implicit parentheses inserted:

\snip{RankN}{forall3}

Here we can see that indeed the \hs{forall a.} \emph{is} to the left of
a function arrow---the outermost one. And so, the rank of a function is simply
the number of arrows its deepest \hs{forall} is to the left of.

\begin{exercise}
What is the rank of \ty{Int -> forall a. a -> a}? Hint: try adding the
explicit parentheses.
\end{exercise}
\begin{solution}
  \ty{Int -> forall a. a -> a} is rank-1.
\end{solution}

\begin{exercise}
What is the rank of \ty{(a -> b) -> (forall c. c -> a) -> b}? Hint:
recall that the function arrow is right-associative, so \ty{a -> b -> c} is
actually parsed as \ty{a -> (b -> c)}.
\end{exercise}
\begin{solution}
  \ty{(a -> b) -> (forall c. c -> a) -> b} is rank-2.
\end{solution}

\begin{exercise}
What is the rank of \ty{((forall x. m x -> b (z m x)) -> b (z m a)) ->
m a}? Believe it or not, this is a real type signature we had to write back in
the bad old days before \ty{MonadUnliftIO}!
\end{exercise}
\begin{solution}
  Rank-3.
\end{solution}


\section{The Continuation Monad}

An interesting fact is that the types \ty{a} and \ty{forall r. (a -> r) -> r}
are \glslink{isomorphism}{isomorphic}. This is witnessed by the following
functions:

\snip{RankN}{cont}
\snip{RankN}{runCont}

Intuitively, we understand this as saying that having a value is just as good as
having a function that will give that value to a callback. Spend a few minutes
looking at \hs{cont} and \hs{runCont} to convince yourself you know why these
things form an isomorphism.

The type \ty{forall r. (a -> r) -> r} is known as being in
\defn{continuation-passing style} or more tersely as \defn{CPS}.

Recall that isomorphisms are transitive. If we have an isomorphism \iso{t1}{t2},
and another \iso{t2}{t3}, we must also have one \iso{t1}{t3}.

Since we know that \iso{Identity a}{a} and that \iso{a}{forall r. (a -> r) ->
r}, we should expect the transitive isomorphism between \ty{Identity a} and CPS.
Since we know that \ty{Identity a} is a \ty{Monad} and that isomorphisms
preserve typeclasses, we should expect that CPS also forms a \ty{Monad}.

We'll use a newtype as something to attach this instance to.

\snip{RankN}{Cont}

\begin{exercise}
Provide a \ty{Functor} instance for \ty{Cont}. Hint: use lots of type holes, and
  an explicit lambda whenever looking for a function type. The implementation is
  sufficiently difficult that trying to write it point-free will be particularly
  mind-bending.
\end{exercise}
\begin{solution}
\unspacedSnip{RankN}{contFunctor}
\end{solution}

\begin{exercise}
Provide the \ty{Applicative} instances for \ty{Cont}.
\end{exercise}
\begin{solution}
\unspacedSnip{RankN}{contApplicative}
\end{solution}

\begin{exercise}
Provide the \ty{Monad} instances for \ty{Cont}.
\end{exercise}
\begin{solution}
\unspacedSnip{RankN}{contMonad}
\end{solution}

One of the big values of \ty{Cont}'s \ty{Monad} instance is that it allows us to
flatten JavaScript-style ``pyramids of doom.''

For example, imagine the following functions all perform asynchronous \ty{IO} in
order to compute their values, and will call their given callbacks when
completed.

\snip{RankN}{withVersionNumber}
\snip{RankN}{withTimestamp}
\snip{RankN}{withOS}

We can write a ``pyramid of doom''-style function that uses all three callbacks
to compute a value:

\snip{RankN}{releaseString}

Notice how the deeper the callbacks go, the further indented this code becomes.
We can instead use the \ty{Cont} (or \ty{ContT} if we want to believe these
functions are actually performing \ty{IO}) to flatten this pyramid.

\snip{RankN}{releaseStringCont}

When written in continuation-passing style, \hs{releaseStringCont} hides the
fact that it's doing nested callbacks.

\begin{exercise}
There is also a monad transformer version of \ty{Cont}. Implement it.
\end{exercise}
\begin{solution}
\unspacedSnip{RankN}{ContT}

  The \ty{Functor}, \ty{Applicative} and \ty{Monad} instances for \ty{ContT} are
  identical to \ty{Cont}.
\end{solution}

\end{document}

