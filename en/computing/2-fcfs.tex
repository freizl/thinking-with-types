\documentclass[book.tex]{subfiles}
\begin{document}

\chapter{First Class Families}

\section{Defunctionalization}

\preamble{Defunc}

Until recently, it was believed that type families had no chance of being first
class; because they're unable to be partially applied, reuse and abstraction
seemed impossible goals. Every type-level \hs{fmap} would need to be specialized
with its mapping function built-in, and there seemed no way of abstracting away
this repetitive boilerplate.

The work of \lyxia{} \cite{fcf} has relaxed this limitation, not by providing
unsaturated type families, but by giving us tools for working around them. It
works via \defnn{Defunctionalization}{defunctionalization}---the process of
replacing an instantiation of a polymorphic function with a specialized label
instead.

For example, rather than the function \hs{fst}:

\snip{Defunc}{fst}

We can instead defunctionalize it by providing an equivalent label:

\snip{Defunc}{Fst}

All that's left to implement is the actual evaluation function. This can be
codified as a typeclass with a functional dependency to guide the return type.

\snip{Defunc}{Eval}

The syntax \hs{| l -> t} at \ann{1} is known as a \defn{functional dependency},
and states that the type \ty{t} is fully determined by the type \ty{l}. Here,
\ty{l} is the type of our defunctionalized label, and \ty{t} is the return type
of the evaluation.

\snip{Defunc}{EvalFst}

Despite being roundabout, this approach works just as well as \hs{fst} itself
does.

\begin{dorepl}{Defunc}
eval (Fst ("hello", True))
\end{dorepl}

\begin{exercise}
Defunctionalize \hs{listToMaybe :: [a] -> Maybe a}.
\end{exercise}
\begin{solution}
\unspacedSnip{Defunc}{ListToMaybe}
\snip{Defunc}{EvalListToMaybe}
\end{solution}

Even higher-order functions can be defunctionalized:

\snip{Defunc}{Map}

The \ty{dfb} type at \ann{1} could be labeled simply as \ty{b}, but we will
enforce it is a defunctionalized symbol---evaluation of \ty{MapList} will depend on
an \ty{Eval dfb} instance. This name helps suggest that.

\snip{Defunc}{EvalMap}

Pay attention to \ann{1}---rather than consing \hs{f a} to the mapped list, we
instead cons \hs{eval (f a)}. While there is morally no difference, such an
approach allows defunctionalization of \hs{map} to propagate evaluation of other
defunctionalized symbols. We can see it in action:

\begin{dorepl}{Defunc}
eval (MapList Fst [("hello", 1), ("world", 2)])
\end{dorepl}


\section{Type-Level Defunctionalization} \label{fcf}

\preamble{FCTF}

This entire line of reasoning lifts, as Xia shows, to the type-level where it
fits a little more naturally. Because type families are capable of
discriminating on types, we can write a defunctionalized symbol whose type
corresponds to the desired type-level function.

These are known as \defnn{first class families}{first class family}, or
\defn{FCF}s for short.

We begin with a kind synonym, \ty{Exp a} which describes a type-level function
which, when evaluated, will produce a type of kind \kind{a}.

\snip{FCTF}{Exp}

This ``evaluation'' is performed via an open type family \ty{Eval}. \ty{Eval}
matches on \ty{Exp a}s, mapping them to an \kind{a}.

\snip{FCTF}{Eval}

To write defunctionalized ``labels'', empty data-types can be used. As an
illustration, if we wanted to lift \hs{snd} to the type-level, we write a
data-type whose kind mirrors the type of \hs{snd}.

\snip{FCTF}{Snd}

\begin{dorepl}{FCTF}
:t snd
:kind Snd
\end{dorepl}

The type of \hs{snd} and kind of \ty{Snd} share a symmetry, if you ignore the
trailing \hs{-> Type} on \ty{Snd}. An instance of \ty{Eval} can be used to
implement the evaluation of \ty{Snd}.

\snip{FCTF}{EvalSnd}

\begin{dorepl}{FCTF}
:kind! Eval (Snd '(1, "hello"))
\end{dorepl}

Functions that perform pattern matching can be lifted to the defunctionalized
style by providing multiple type instances for \ty{Eval}.

\snip{FCTF}{FromMaybe}

\ty{Eval} of \ty{FromMaybe} proceeds as you'd expect.

\begin{dorepl}{FCTF}
:kind! Eval (FromMaybe "nothing" ('Just "just"))
:kind! Eval (FromMaybe "nothing" 'Nothing)
\end{dorepl}

\begin{exercise}
Defunctionalize \hs{listToMaybe} at the type-level.
\end{exercise}
\begin{solution}
\unspacedSnip{FCTF}{ListToMaybe}
\end{solution}

However, the real appeal of this approach is that it allows for
\emph{higher-order} functions. For example, \hs{map} is lifted by taking a
defunctionalization label of kind \kind{a -> Exp b} and applying it to a
\kind{[a]} to get a \kind{Exp [b]}.

\snip{FCTF}{MapList}

\ty{Eval} of \ty{'[]} is trivial:

\snip{FCTF}{EvalNil}

And \ty{Eval} of cons proceeds in the only way that will type check, by
evaluating the function symbol applied to the head, and evaluating the
\ty{MapList} of the tail.

\snip{FCTF}{EvalCons}

Behold! A type-level \ty{MapList} is now \emph{reusable}!

\begin{dorepl}{FCTF}
:kind! Eval (MapList (FromMaybe 0) ['Nothing, ('Just 1)])
:kind! Eval (MapList Snd ['(5, 7), '(13, 13)])
\end{dorepl}

\begin{exercise}
Defunctionalize \hs{foldr ::} \\ \ty{(a -> b -> b) -> b -> [a] -> b}.
\end{exercise}
\begin{solution}
\unspacedSnip{FCTF}{Foldr}
\end{solution}


\section{Working with First Class Families}

Interestingly, first-class families form a monad at the type-level.

\snip{FCTF}{Pure}
\snip{FCTF}{bind}

As such, we can compose them in terms of their Kleisli composition.
Traditionally the fish operator (\hs{<=<}) is used to represent this combinator
in Haskell.  We are unable to use the more familiar period operator at the
type-level, as it conflicts with the syntax of the \ty{forall} quantifier.

\snip{FCTF}{kleisli}

\begin{dorepl}{FCTF}
:set -XPolyKinds
type Snd2 = Snd <=< Snd
:kind! Eval (Snd2 '(1, '(2, 3)))
\end{dorepl}

While \ty{(<=<)} at the type-level acts like regular function composition
\hs{(.)}, \ty{(=<<)} behaves like function application \hs{(\$)}.

\begin{dorepl}{FCTF}
:kind! Eval (Snd <=< FromMaybe '(0, 0) =<< Pure (Just '(1, 2)))
\end{dorepl}

The \pkg{first-class-families}\cite{fcf} package provides most of \ty{Prelude}
as FCFs, as well as some particularly useful functions for type-level
programming. For example, we can determine if any two types are the
same---remember, there is no \ty{Eq} at the type-level.

\snip{FCTF}{TyEq}
\snip{FCTF}{EvalTyEq}
\snip{FCTF}{TyEqImpl}

We also have the ability to collapse lists of \kind{Constraint}s.

\snip{FCTF}{Collapse}

Which leads us very naturally to:

\snip{FCTF}{All}
\snip{FCTF}{Pure1}

And we find ourselves with a much nicer implementation of \ty{All} than we wrote
in chapter 5.

%\todo{index}

\begin{dorepl}{FCTF}
:kind! Eval (All Eq '[Int, Bool])
\end{dorepl}


\section{Ad-Hoc Polymorphism}

Consider the following definition of \ty{Map}.

\snip{FCTF}{Map}

Because type families are capable of discriminating on types, we can write
several instances of \ty{Eval} for \ty{Map}. For example, for lists:

\snip{FCTF}{Map4List}

And also for \ty{Maybe}:

\snip{FCTF}{Map4Maybe}

Why not an instance for \ty{Either} while we're at it.

\snip{FCTF}{Map4Either}

As you might expect, this gives us ad-hoc polymorphism for a promoted \hs{fmap}.

\begin{dorepl}{FCTF}
:kind! Eval (Map Snd ('Just '(1, 2)))
:kind! Eval (Map Snd '[ '(1, 2) ])
:kind! Eval (Map Snd (Left 'False))
\end{dorepl}

\begin{exercise}
Write a promoted functor instance for tuples.
\end{exercise}
\begin{solution}
\unspacedSnip{FCTF}{MapTuple}
\end{solution}

This technique of ad-hoc polymorphism generalizes in the obvious way, allowing
us to implement the semigroup operation \ty{Mappend}:

\snip{FCTF}{Mappend}

However, at first glance, it's unclear how the approach can be used to implement
\ty{Mempty}. Type families are not allowed to discriminate on their return type.
We can cheat this restriction by muddying up the interface a little and making
the ``type application'' explicit. We give the \gls{kind signature} of
\ty{Mempty} as follows:

\snip{FCTF}{Mempty}

The understanding here is that given a type of any monoidal kind \kind{k},
\ty{Mempty} will give back the monoidal identity for that kind. This can be done
by matching on a rigid kind signature, as in the following instances.

\snip{FCTF}{mempties}

\end{document}
