\documentclass[book.tex]{subfiles}
\begin{document}

\chapterNoTOC{Preface}

Thinking with Types started, as so many of my projects do, accidentally. I was
unemployed, bored, and starting to get tired of answering the same questions
over and over again in Haskell chat-rooms. And so I started a quick document,
jotting down a bunch of type-level programming topics I thought it'd be fun to
write blog posts about.

This document rather quickly turned into an outline of what those blog posts
might look like, but as I was about to tease it apart into separate files I
stopped myself. Why not turn it into a book instead?

I approached some friends to see if anyone was interested in writing it with me.
A few nibbles, but nobody had time they wanted to dedicate to such a thing. My
excitement subsequently burned out, and the idea lay dormant on the
back-burner for a few months.

But I was still unemployed, and I was still bored, and I found myself slowly
fleshing out chapters regardless. My enthusiasm for \emph{writing a book} had
died down, but I still felt the urge to \emph{write.} A friend caught me writing
one day, and dared me to publish what I had. I acquiesced.

And so on July 8th, 2018, I posted a 37 page document to reddit, gauging if
there was any interest from the community in such a book. To my continual
surprise, there was. The response was about 100x bigger than I was expecting.
Kind words and letters of support rolled in, many of whom promised to pay me in
order to continue writing it.

That was enough for me. I put together a Patreon, started selling early access
to the book, and was off to the races. The promise was to publish weekly
updates, which---combined with not wanting to commit fraud---kept me extremely
motivated to get this book finished. It's a powerful technique to stay focused,
and I'd strongly recommend it to anyone who is better at starting projects than
finishing them.

It sounds cliche, but this book couldn't have happened without the overwhelming
support of the Haskell community. It's particularly telling that every day I
learn new things from them about this marvelous language, even after five years.

Written with love by Sandy Maguire. 2018.

\end{document}

